% !TEX TS-program = lualatex
% !TeX spellcheck = ru_RU
% !TEX root = abstract-machines.tex
\documentclass[aspectratio=1610]{beamer}


\makeatletter
\appto\input@path{{libs/awesome-beamer}, {libs/smile}}
\makeatother

\definecolor{teal}{HTML}{008080}
\usetheme[english, color, coloraccent=teal, listings, footnote, secslide]{awesome}


\usepackage{polyglossia}
\setdefaultlanguage{russian}
\setotherlanguage{english}

\usepackage[
  backend=biber,
  style=alphabetic
]{biblatex}
\usepackage[strict,autostyle]{csquotes}
\nocite{*}
\addbibresource{refs.bib}

\usepackage{fontawesome5}
\def\ergo{\raisebox{.5pt}{\scalebox{.8}{\faCaretRight}}}

\usepackage{contour}
\usepackage{tikzducks}

\usepackage{fontspec}
\setmainfont{CMU Serif}
\setsansfont[
  Ligatures=TeX,
  BoldFont={* Medium},
]{Fira Sans}
\setmonofont[
  Path = ./fonts/,
  Scale = .9,
  Extension = .ttf,
  Contextuals=Alternate,
  BoldFont={*-Bold},
  UprightFont={*-Regular},
]{Fira Code}


\tikzset{bb/.style={draw=tcbcolframe,dash pattern=on 1mm off 1mm,dash phase=0.5mm,tcb@spec,segmentation@style}}
\def\setlinetext#1{\small\color{tcbcolframe}\contourlength{1.5pt}\contour{tcbcolback}{#1}}
% \renewenvironment{block}[1][]{%
% 	\begin{beamerbox}[segmentation code={
% 					\path[bb] (segmentation.west) to node{\setlinetext{#1}} (segmentation.east);
% 				}]{gray}{}%
% 		}{\end{beamerbox}}
\newenvironment{sblock}[1][]{
  \begin{beamerbox}[sidebyside,segmentation code={
          \path[bb] (segmentation.north) to node[rotate=90]{\setlinetext{#1}} (segmentation.south);
        }]{gray}{}%
    }{\end{beamerbox}}

\usepackage{emoji}



% Some label on the background picture
% \addtobeamertemplate{title page}{}{
% 	\tikz[o]\node[anchor=south east,outer sep=0pt] at (current page.south east)
% 	{\fontsize{4}{4}\selectfont\color{white}This image was generated by AI (DALLE 3)};
% }

\usepackage{multicol}

\usepackage{datetime}
\renewcommand{\dateseparator}{.}

\usepackage[verbatim]{lstfiracode}
\lstdefinestyle{firastyleb}{style=FiraCodeStyle,style=smile@lst@base}
\lstdefinestyle{firastylep}{style=FiraCodeStyle,style=smile@lst@plain}
\lstset{
  tabsize=2,
  style=firastylep,
  %apptoliterate={{=> }{{{=> }}}3} % for whatever reason, the space would otherwise be eaten
}
\lstdefinelanguage{myhaskell}{
  language=haskell,
}
\newcommand\hn[2][]{\lstinline[language=myhaskell,#1]{#2}}
\def\inlineblock#1{\tikz[anchor=base,baseline]\node[inner sep=0.3333em,rnd,fill=lightgray!50,anchor=base,baseline=] {#1};}
\newcommand\h[2][]{\inlineblock{\hn[#1]{#2}}}
\makeatletter
\lstnewenvironment{haskell}{\lstset{language=myhaskell}}{}
\lstnewenvironment{xshaskell}{\lstset{language=myhaskell,basicstyle=\smile@lst@style@base\footnotesize}}{}
\lstnewenvironment{xxshaskell}{\lstset{language=myhaskell,basicstyle=\smile@lst@style@base\tiny}}{}
\def\inpuths #1 from #2 to #3;{\lstinputlisting[language=myhaskell,firstline=#2,lastline=#3]{#1}}
\def\inputhsxs #1 from #2 to #3;{\lstinputlisting[language=myhaskell,firstline=#2,lastline=#3,basicstyle=\smile@lst@style@base\footnotesize]{#1}}
\def\inputhsxxs #1 from #2 to #3;{\lstinputlisting[language=myhaskell,firstline=#2,lastline=#3,basicstyle=\smile@lst@style@base\tiny]{#1}}
\makeatother

\usepackage{booktabs}
% \usepackage{arydshln}
% \usepackage[utf8]{inputenc}
\usepackage{array}
\usepackage{colortbl}
\usepackage{xcolor}
\usepackage{caption}
\usepackage{amsmath}
\usepackage{mathtools}
\usepackage{emoji}
\usepackage{appendix}

\definecolor{myPurple}{HTML}{4D0694}
\definecolor{myRed}{HTML}{940606}
\background{pictures/magicstudio-art.jpg}
\title{Абстрактные машины}
\subtitle{Реализация функциональных языков программирования}
\author{Ефим Кубышкин}
\email{efimkub@mail.ru}
\institute{Санкт-Петербургский Государственный Университет}
\uni{СПбГУ}
\location{СПБ}
\date{\ddmmyyyydate\today}

\begin{document}
\maketitle

\section{Введение}

% \begin{frame}
%   \frametitle[Что и зачем]{Абстрактные машины}
%   Абстрактная машина~--- абстракция
% \begin{itemize}
%   \item Что такое абстрактная машина и её роль в компиляции и интерпретации.
%   \item Различие между абстрактными машинами и реальными архитектурами.
%   \item Основные компоненты:
%         \begin{itemize}
%           \item Хранение состояния (стек, куча).
%           \item Выполнение инструкций.
%           \item Управление памятью и вычислениями.
%         \end{itemize}
% \end{itemize}
% \end{frame}
\begin{frame}{Что такое абстрактная машина?}
  \textbf{Абстрактная машина}~--- теоретическая модель вычислительной системы

  \begin{itemize}
    \item Не такое низкоуровневое как компиляция, не такое высокоуровневое, как интерпретация
    \item[$\Rightarrow$] Можно делать более или менее низкоуровневые, но машинно-независимые оптимизации
  \end{itemize}
  Придумать свою абстрактную машину можно легко, нужно придумать
  \begin{itemize}
    \item Представление памяти для промежуточных данных
    \item Инструкции и правила исполнения этих инструкций
  \end{itemize}
\end{frame}

\begin{frame}{Почему важны для функциональных языков?}
  Для каждого языка удобно придумать свою абстрактную машину, чтобы она лучше отражала особенности языка
  \begin{itemize}
    \item Оптимизации, например, хвостовой рекурсии
    \item Порядок вычисления
    \item By design обработка каррированных функций
  \end{itemize}
\end{frame}

\section{Конкретные абстрактные машины}

\subsection{SECD}

\begin{frame}
  \frametitle{SECD машина}
  Состоит из четырёх компонент
  \begin{itemize}
    \item Стек (\textbf{S}tack) для хранения промежуточных результатов
    \item Среда (\textbf{E}nvironment) даёт значения переменным
    \item Последовательность инструкций (\textbf{C}ontrol) для того, чтобы понимать, что ещё осталось исполнить
    \item Дамп (\textbf{D}ump) для хранения состояния возврата
  \end{itemize}
  Две из которых (\textbf{S} и \textbf{D}) будут представлены в одном стеке
\end{frame}



\begin{frame}
  \frametitle{Инструкции}
  \begin{itemize}
    \item[\color{teal}\texttt{ACCESS(N)}] Положить на стек \texttt{N}-е поле окружения
    \item[\color{teal}\texttt{CLOSURE(c)}] Положить на стек замыкание с кодом \texttt{c} и текущим окружением
    \item[\color{teal}\texttt{LET}] Взять значения со стека и добавить к окружению
    \item[\color{teal}\texttt{ENDLET}] Выбросить первую запись из окружения
    \item[\color{teal}\texttt{APPLY}] Достать со стека замыкание и аргумент, выполнить применение
    \item[\color{teal}\texttt{RETURN}] Прекратить выполнение текущей функции, вернуться к
          вызывавшему
  \end{itemize}
\end{frame}

\begin{frame}
  \frametitle{Компиляция}
  \textbf{SECD} работает с $\lambda$-термами в безымянной форме с использованием индексов де~Брёйна
  \begin{wide}
    \begin{align*}
      \mathtt{\mathcal{C}(n)}                                 & = \mathtt{ACCESS(n)}                                  \\
      \mathtt{\mathcal{C}(\lambda a)}                         & = \mathtt{CLOSURE(\mathcal{C}(a); RETURN)}            \\
      \mathtt{\mathcal{C}(\mathbf{let} \ a \ \mathbf{in}\ b)} & = \mathtt{\mathcal{C}(a); LET; \mathcal{C}(b);ENDLET} \\
      \mathtt{a \ b}                                          & = \mathtt{\mathcal{C}(a); \mathcal{C}(b); APPLY}
    \end{align*}
  \end{wide}
\end{frame}

\begin{frame}[fragile]
  \frametitle{Иполнение инструкций}
  \begin{wide}

    \begin{table}
      \centering
      \begin{tabular}{@{}>{\color{teal}\ttfamily}l>{\color{myPurple}\ttfamily} c >{\color{myRed}\ttfamily}r !{\vline height 1.5em depth 0.7em width 2pt} >{\color{teal}\ttfamily}l >{\color{myPurple}\ttfamily}c >{\color{myRed}\ttfamily}r@{}}
        \multicolumn{3}{c!{\vline width 2pt}}{Machine state before} & \multicolumn{3}{c}{Machine state after}                                      \\
        Code                                                        & Env                                     & Stack      & Code & Env  & Stack   \\
        \hline
        LET; c                                                      & e                                       & v.s        & c    & v.e  & s       \\
        \hline
        ENDLET; c                                                   & v.e                                     & s          & c    & e    & s       \\
        \hline
        ACCESS(N); c                                                & e                                       & s          & c    & e    & e(N).s  \\
        \hline
        CLOSURE(c'); c                                              & e                                       & s          & c    & e    & c'[e].s \\
        \hline
        APPLY; c                                                    & e                                       & v.c'[e'].s & c'   & v.e' & c[e].s  \\
        \hline
        RETURN; c                                                   & e                                       & v.c'[e].s  & c'   & e'   & v.s     \\
        \hline
      \end{tabular}
    \end{table}
  \end{wide}
\end{frame}

\begin{frame}[fragile]
  \frametitle{Пример}
  \begin{wide}
    $(\lambda x. (x + 1)) 2  \rightsquigarrow \mathtt{CLOSURE(ACCESS(1);CONST(1);ADD;RETURN);CONST(2);APPLY}$
    \begin{table}
      \centering
      \begin{tabular}{@{}>{\color{teal}\ttfamily}l !{\vline height 1.4em} >{\color{myPurple}\ttfamily} c >{\color{myRed}\ttfamily} r }
        Code                          & Env & Stack               \\ \hline
        CLOSURE(c); CONST(2); APPLY   & []  & $\varnothing$       \\
        CONST(2); APPLY               & []  & c[]                 \\
        APPLY                         & []  & 2.c[]               \\
        ACCESS(1);CONST(1);ADD;RETURN & 2   & $\varepsilon$[]     \\
        CONST(1); ADD; RETURN         & 2   & 2.$\varepsilon$[]   \\
        ADD; RETURN                   & 2   & 1.2.$\varepsilon$[] \\
        RETURN                        & 2   & 3.$\varepsilon$[]   \\
        $\varepsilon$                 & []  & 3
      \end{tabular}
    \end{table}
  \end{wide}
\end{frame}

\begin{frame}
  \frametitle{Особенности}
  \begin{itemize}
    \item Очень простая
    \item \texttt{Call-by-value}
    \item Беды с каррированными функциями
    \item Можно подправить совсем чуть-чуть и будет обработка хвостовой рекурсии
  \end{itemize}
\end{frame}
\subsection{Кривин}
\begin{frame}
  \frametitle{Машина Кривина}
  Всё то же самое, что и у \textbf{SECD}
  \begin{itemize}
    \item Указатель на код
    \item Окружение
    \item Стек промежуточных результатов
  \end{itemize}
  Однако стек и окружение теперь содержат \textbf{thunks}: замыкания, которые не будут вычисляться до тех пор, пока не понадобятся.

  Так что да, главная особенность~--- \texttt{call-by-name}
\end{frame}

\begin{frame}
  \frametitle[И схема компиляции]{Инструкции}
  \begin{itemize}
    \item[\color{teal}\texttt{ACCESS(N)}] Начать вычислять \texttt{N}-ый \textbf{thunk}
    \item[\color{teal}\texttt{PUSH(c)}] Положить \textbf{thunk} на стек с кодом \texttt{c}
    \item[\color{teal}\texttt{GRAB}] Переложить аргумент из стека в окружение
  \end{itemize}
  \begin{align*}
    \mathtt{\mathcal{C}(n)}         & = \mathtt{ACCESS(n)}                            \\
    \mathtt{\mathcal{C}(\lambda a)} & = \mathtt{GRAB;\mathcal{C}(a)}                  \\
    \mathtt{\mathcal{C}(a \ b)}     & = \mathtt{PUSH(\mathcal{C}(b)); \mathcal{C}(a)}
  \end{align*}
\end{frame}
\begin{frame}
  \frametitle{Исполнение инструкций}
  \begin{absolutewide}

    \begin{table}
      \centering
      \begin{tabular}{@{}>{\color{teal}}l>{\color{myPurple}} c >{\color{myRed}}r !{\vline height 1.5em depth 0.7em width 2pt} >{\color{teal}}l >{\color{myPurple}}c >{\color{myRed}}r@{}}
        \multicolumn{3}{c!{\vline width 2pt}}{Machine state before} & \multicolumn{3}{c}{Machine state after}                                                                               \\
        Code                                                        & Env                                     & Stack    & Code & Env      & Stack                                          \\
        \hline
        ACCESS(N); c                                                & e                                       & s        & c'   & e'       & {\color{black}\lstinline{if e(N) == c'[e']}} s \\
        \hline
        GRAB; c                                                     & e                                       & c'[e'].s & c    & c'[e'].e & s                                              \\
        \hline
        PUSH(c'); c                                                 & e                                       & s        & c    & e        & c'[e].s                                        \\
        \hline
      \end{tabular}
    \end{table}
  \end{absolutewide}
  Начальное состояние: $\langle \mathtt{C(a)}, \varnothing , \varepsilon\rangle$

  Конечное состояние:  $\langle \mathtt{GRAB;c}, e, \varepsilon \rangle$

\end{frame}

\begin{frame}
  \frametitle{Пример}
  \begin{wide}
    $(\lambda y . 1) ((\lambda x. x x)(\lambda x. x x)) \rightsquigarrow ((\lambda \mathbf{1}))((\lambda (1 \ 1))(\lambda (1 \ 1))) \rightsquigarrow \alt<2->{\mathtt{PUSH({\color{myRed} c_1}); GRAB; CONST(\mathbf{1})}}{\\ \rightsquigarrow \mathtt{PUSH(P(G; P(A(1)); A(1)); G; P(A(1); A(1))); GRAB; CONST(\mathbf{1})}}$
    \onslide<2->{\begin{table}
        \centering
        \begin{tabular}{@{}>{\color{teal}\ttfamily}l !{\vline height 1.4em} >{\color{myPurple}\ttfamily} c >{\color{myRed}\ttfamily} c }
          Code                        & Env     & Stack         \\ \hline
          PUSH($c_1$); GRAB; CONST(1) & []      & $\varnothing$ \\
          GRAB; CONST(1)              & []      & $c_1$[]       \\
          CONST(1)                    & $c_1$[] & $\varnothing$ \\
          $\varepsilon$               & $c_1$[] & $1$
        \end{tabular}
      \end{table}}
  \end{wide}
\end{frame}

\begin{frame}
  \frametitle{На практике}
  \begin{itemize}
    \item Нужно думать о том, что некоторые операции, например, сложение всё-таки надо сделать строгими
    \item Нужно думать о том, чтобы одинаковые подвыражения не высчитывались многократно
  \end{itemize}
  Короче, для \textsc{Haskell} машина Кривина не подойдёт
\end{frame}

\subsection{ZAM}
\begin{frame}
  \frametitle{ZINC abstract machine}
  \textbf{ZINC} (\textbf{Z}INC \textbf{I}s \textbf{N}ot \textbf{C}aml)~--- модель исполнения Caml Light и \texttt{OCaml}, скомпилированных в переносимый байткод.
  \begin{itemize}
    \item Подходит для каррированных функций
    \item \texttt{Call-by-value}
  \end{itemize}
\end{frame}

\begin{frame}%%поправить
  \frametitle{Eval-apply и push-enter}
  Как вычисляется функция $f \ a$
  \begin{itemize}
    \item[Eval-apply] В \textbf{SECD} $\beta$-редукция выполняется \enquote{вызвавшей} стороной
    \item[Push-enter] В машине Кривина $\beta$-редукция выполняется самой функцией
  \end{itemize}
  \textbf{ZAM} в некотором роде объединяет \textbf{SECD} и машину Кривина

\end{frame}

\begin{frame}
  \frametitle[наглядно]{Eval-apply и push-enter}
  \begin{wide}
    Допустим $\mathtt{f \ a_1 \ a_2}$, где $\mathtt{f = \lambda.\lambda.b}$
  \end{wide}
  \begin{absolutewide}
    \begin{center}
      \begin{tabular}{c|c}
        \textbf{Eval-apply} & \textbf{Push-enter} \\ \hline
        \begin{tabular}[t]{@{}l@{}}
          \texttt{eval} $\mathtt{f}$                                       \\
          \texttt{eval} $\mathtt{a_1}$                                     \\
          {\color{teal}\texttt{APPLY}}                                     \\
          \hspace{3em}\rotatebox{-45}{$\rightarrow$}                       \\
          \hspace{5em}{\color{teal}\texttt{CLOSURE}($\mathtt{\lambda.b}$)} \\
          \hspace{5em}{\color{teal}\texttt{RETURN}}                        \\
          \hspace{3em}\rotatebox{45}{$\leftarrow $}                        \\
          \texttt{eval} $a_2$                                              \\
          {\color{teal}\texttt{APPLY}}                                     \\
          \hspace{3em}\rotatebox{-45}{$\rightarrow$}                       \\
          \hspace{5em}\texttt{eval} $b$
        \end{tabular}
                            &
        \begin{tabular}[t]{@{}l@{}}
          \texttt{push} $\mathtt{a_2}$               \\
          \texttt{push} $\mathtt{a_1}$               \\
          \texttt{find \& enter} $\mathtt{f}$        \\
          \hspace{8em}\rotatebox{-45}{$\rightarrow$} \\
          \hspace{10em} {\color{teal}\texttt{GRAB}}  \\
          \hspace{10em} {\color{teal}\texttt{GRAB}}  \\
          \hspace{10em} \texttt{eval} $\mathtt{b}$
        \end{tabular}
      \end{tabular}
    \end{center}
  \end{absolutewide}
\end{frame}

\begin{frame}
  \frametitle{Инструкции}
  \begin{wide}
    \texttt{ACCESS(n), CLOSURE(C), ENDLET} такие же, как и в \textbf{SECD}
  \end{wide}
  \begin{itemize}
    \item[\color{teal}\texttt{RETURN}] Прекратить выполнение текущей функции, вернуться к вызывавшему (как в \textbf{SECD}).
          Или запомнить посчитанное на стеке значение и вернуться к вызывавшему
    \item[\color{teal}\texttt{GRAB}] Переложить аргумент со стека в окружение (Кривин) или положить на стек частично применённую функцию
    \item[\color{teal}\texttt{APPLY}] Достать со стека замыкание (даже проще, чем в \textbf{SECD})
    \item[\color{teal}\texttt{PUSHRETADDR}] Положить на стек функцию и пометить точку применения функции
  \end{itemize}
\end{frame}

\begin{frame}
  \frametitle{Компиляция}
  $\mathcal{T}$ для выражений в хвостовой рекурсии
  \begin{wide}
    \begin{align*}
      \mathcal{T}\mathtt{(\lambda a)}                         & = \mathtt{GRAB;\mathcal{T}(a)}                                                         \\
      \mathcal{T}\mathtt{(\mathbf{let}\ a \ \mathbf{in} \ b)} & = \mathtt{\mathcal{C}(a);GRAB;\mathcal{T}(b)}                                          \\
      \mathcal{T}\mathtt{(a\ a_1 \dots a_n) }                 & = \mathtt{\mathcal{C}(a_n);\dots; \mathcal{C}(a_1);\mathcal{T}(a)}                     \\
      \mathcal{T}\mathtt{(a)}                                 & = \mathtt{\mathcal{C}(a); RETURN}                                                      \\
      \\
      \mathcal{C}\mathtt{(n)}                                 & = \mathtt{ACCESS(n)}                                                                   \\
      \mathcal{C}\mathtt{(\lambda a)}                         & = \mathtt{CLOSURE(GRAB;\mathcal{T}(a))}                                                \\
      \mathcal{C}\mathtt{(\mathbf{let}\ a\ \mathbf{in}\ b)}   & = \mathtt{\mathcal{C}(a);GRAB;\mathcal{C}(b);ENDLET}                                   \\
      \mathcal{C}\mathtt{(a\ a_1 \dots a_n)}                  & = \mathtt{PUSHRETADDR(k);\mathcal{C}(a_n);\dots;\mathcal{C}(a_1);\mathcal{C}(a);APPLY} \\
    \end{align*}
  \end{wide}
  Здесь $\mathtt{k}$  обозначает код, который идёт после \texttt{APPLY}
\end{frame}

\begin{frame}
  \frametitle{Выполнение инструкций}
  \begin{wide}
    \begin{table}
      \centering
      \begin{tabular}{@{}>{\color{teal}\ttfamily}l>{\color{myPurple}\ttfamily} c >{\color{myRed}}r !{\vline height 1.5em depth 0.7em width 2pt} >{\color{teal}\ttfamily}l >{\color{myPurple}\ttfamily}c >{\color{myRed}}r@{}}
        \multicolumn{3}{c!{\vline width 2pt}}{Machine state before} & \multicolumn{3}{c}{Machine state after}                                                       \\
        Code                                                        & Env                                     & Stack               & Code & Env & Stack            \\
        \hline
        GRAB; c                                                     & e                                       & v.s                 & c    & v.e & s                \\%% перекладываем значение со стека в окружение 
        \hline
        GRAB; c                                                     & e                                       & $\square$.c'.e'.s   & c'   & e'  & (GRAB;c)[e].s    \\%% кладём на стек частично применённую функцию
        \hline
        RETURN; c                                                   & e                                       & v.$\square$.c'.e'.s & c'   & e'  & v.s              \\%% даём посчитанные аргументы вызываемой функции
        \hline
        RETURN; c                                                   & e                                       & c'[e'].s            & c'   & e'  & s                \\%% возвращаем сразу результат и продолжаем вычисление (tail-apply)
        \hline
        PUSHRETADDR(c'); c                                          & e                                       & s                   & c    & e   & $\square$.c'.e.s \\%% помечаем точку входа функции, запоминаем текущее окружение
        \hline
        APPLY; c                                                    & e                                       & c'[e'].s            & c'   & e'  & s                \\%% передаёт контроль выполнения замыканию с посчитанным значением
        \hline
      \end{tabular}
    \end{table}
  \end{wide}
\end{frame}

\begin{frame}
  \frametitle[Полное применение]{Пример}
  \begin{wide}
    $\mathtt{(\lambda.\lambda.\lambda.\mathbf{2})(1)(2)(3)}\rightsquigarrow$ \texttt{PUSHRETADDR($\varepsilon$);C(3);C(2);C(1);CLOSURE(a);APPLY}

    где $\mathtt{a} = $ \texttt{GRAB; GRAB; GRAB; ACCESS($\mathbf{2}$); RETURN}
  \end{wide}
  \begin{absolutewide}

    \begin{table}
      \centering
      \begin{tabular}{@{}>{\color{teal}\ttfamily}l !{\vline height 1.2em} >{\color{myPurple}\ttfamily} c >{\color{myRed}\ttfamily} r }
        Code                                                            & Env   & Stack                              \\ \hline
        PUSHRETADDR($\varepsilon$);C(3);C(2);C(1);CLOSURE(a);APPLY      & []    & $\varnothing$                      \\
        C(3);C(2);C(1);CLOSURE(a);APPLY                                 & []    & $\square.\varepsilon.$[]           \\
        C(2);C(1);CLOSURE(a);APPLY                                      & []    & 3.$\square.\varepsilon.$[]         \\
        C(1);CLOSURE(a);APPLY                                           & []    & 2.3.$\square.\varepsilon.$[]       \\
        CLOSURE(a);APPLY                                                & []    & 1.2.3.$\square.\varepsilon.$[]     \\
        APPLY                                                           & []    & a[].1.2.3.$\square.\varepsilon.$[] \\
        a                                                               & []    & 1.2.3.$\square.\varepsilon.$[]     \\
        {\color{black}$\times3$GRAB $\to$} ACCESS($\mathbf{2}$); RETURN & 3.2.1 & $\square.\varepsilon.$[]           \\
        RETURN                                                          & 3.2.1 & 2.$\square.\varepsilon.$[]         \\
        $\varepsilon$                                                   & []    & 2
      \end{tabular}
    \end{table}
  \end{absolutewide}
\end{frame}

\begin{frame}
  \frametitle[Частичное применение]{Пример}
  \begin{wide}
    $\mathtt{(\lambda.\lambda.\lambda.\mathbf{2})(1)(2)}\rightsquigarrow$ \texttt{PUSHRETADDR($\varepsilon$);C(2);C(1);CLOSURE(a);APPLY}

    где $\mathtt{a} = $ \texttt{GRAB; GRAB; GRAB; ACCESS($\mathbf{2}$); RETURN}
  \end{wide}
  \begin{absolutewide}

    \begin{table}
      \centering
      \begin{tabular}{@{}>{\color{teal}\ttfamily}l !{\vline height 1.2em} >{\color{myPurple}\ttfamily} c >{\color{myRed}\ttfamily} r }
        Code                                                                & Env & Stack                            \\ \hline
        PUSHRETADDR($\varepsilon$);C(2);C(1);CLOSURE(a);APPLY               & []  & $\varnothing$                    \\
        C(2);C(1);CLOSURE(a);APPLY                                          & []  & $\square.\varepsilon.$[]         \\
        C(1);CLOSURE(a);APPLY                                               & []  & 2.$\square.\varepsilon.$[]       \\
        CLOSURE(a);APPLY                                                    & []  & 1.2.$\square.\varepsilon.$[]     \\
        APPLY                                                               & []  & a[].1.2.$\square.\varepsilon.$[] \\
        a                                                                   & []  & 1.2.$\square.\varepsilon.$[]     \\
        {\color{black}$\times2$GRAB $\to$} GRAB;ACCESS($\mathbf{2}$);RETURN & 2.1 & $\square.\varepsilon.$[]         \\
        $\varepsilon$                                                       & []  & b[2.1]
      \end{tabular}
    \end{table}
  \end{absolutewide}
  \begin{wide}
    $\mathtt{b} = $ \texttt{GRAB;ACCESS($\mathbf{2}$)}; RETURN
  \end{wide}
\end{frame}
\subsection{CEK}

\begin{frame}
  \frametitle{CEK машина}
  Состоит из трёх компонент
  \begin{itemize}
    \item Последовательность инструкций (\textbf{С}ontrol) показывает, что осталось исполнить
    \item Окружение (\textbf{E}nvironment) даёт значения переменным
    \item Продолжение (C[\textbf{K}]ontinuation) говорит машине, что делать после того, как кончится \textbf{C}
  \end{itemize}
\end{frame}



\begin{frame}
  \frametitle{Окружение}
  Окружение \textbf{E} сопоставляет переменной значение \texttt{W}, которое может быть либо числом, либо замкнутым $\lambda$-термом.

  Если $\lambda$-терм не замкнутый, то создадим замыкание (\texttt{clos})
  \begin{align*}
    \mathtt{W} \Coloneqq & \quad n                              \\
    \mid                 & \quad \mathtt{clos(\lambda x. M, E})
  \end{align*}
  Здесь \texttt{M}~--- тело, а  \texttt{E}~--- окружение замыкания
\end{frame}

\begin{frame}
  \frametitle{Продолжение}
  Продолжение (\textbf{K})~--- это стек фреймов, где фрейм это
  % \begin{wide}
  \begin{align*}
    \mathtt{F}  \Coloneqq & \quad (\mathtt{W} \ \circ)             \\
    \mid                  & \quad(\circ \ \mathtt{M} \ \mathtt{E})
  \end{align*}
\end{frame}

\begin{frame}
  \frametitle{Команды}
  \begin{absolutewide}
    \begin{table}
      \centering
      \begin{tabular}{@{}>{\color{teal}\ttfamily}l>{\color{myPurple}\ttfamily} c >{\color{myRed}\ttfamily}r !{\vline height 1.5em depth 0.7em width 2pt} >{\color{teal}\ttfamily}l >{\color{myPurple}\ttfamily}c >{\color{myRed}\ttfamily}r@{}}
        \multicolumn{3}{c!{\vline width 2pt}}{Machine state before} & \multicolumn{3}{c}{Machine state after}                                                                                                                                                 \\
        Code                                                        & Env                                     & Stack                                        & Code                           & Env                          & Stack                          \\
        \hline
        x                                                           & e                                       & s                                            & e(x)                           & e                            & s                              \\
        \hline
        $\mathtt{M_1M_2}$                                           & e                                       & s                                            & $\mathtt{M_1}$                 & e                            & $\mathtt{(\circ \ M_2 \ e).s}$ \\
        \hline
        $\lambda x. M$                                              & e                                       & s                                            & $\mathtt{cl(\lambda x. M, e)}$ & e                            & s                              \\
        \hline
        W                                                           & $\mathtt{e_1}  $                        & $\mathtt{(\circ \ M \ e_2).s}$               & M                              & $\mathtt{e_2}$               & $\mathtt{(W \ \circ).s}$       \\
        \hline
        W                                                           & $\mathtt{e_1} $                         & $(\mathtt{cl(\lambda x. M, e_2) \ \circ).s}$ & M                              & $\mathtt{[x \mapsto W].e_2}$ & s                              \\
        \hline
      \end{tabular}
    \end{table}
  \end{absolutewide}
  \begin{wide}
    \begin{itemize}
      \item $e(x)$~--- значение переменной $x$ в окружении $e$
      \item $[x \mapsto W]$~--- присваивание переменной $x$ значение $W$
    \end{itemize}
  \end{wide}
\end{frame}

\begin{frame}
  \frametitle{Пример}
  \begin{absolutewide}
    \begin{table}
      \centering
      \begin{tabular}{@{}>{\color{teal}\ttfamily}l !{\vline height 1.4em} >{\color{myPurple}\ttfamily} c >{\color{myRed}\ttfamily} r }
        Code                                               & Env                          & Stack                                                                    \\ \hline
        $((\lambda x.\lambda y. x)1)2$                     & []                           & $\varnothing$                                                            \\
        $(\lambda x.\lambda y. x)1$                        & []                           & $(\circ \ 2 \ [\ ])$                                                     \\
        $\lambda x.\lambda y. x$                           & []                           & $(\circ \ 1 \ [\ ]).(\circ \ 2 \ [\ ])$                                  \\
        $\mathtt{cl}(\lambda x.\lambda y. x, \varnothing)$ & []                           & $(\circ \ 1 \ [\ ]).(\circ \ 2 \ [\ ])$                                  \\
        $1$                                                & []                           & $(\mathtt{cl}(\lambda x.\lambda y. x, [\ ]) \ \circ).(\circ \ 2 \ [\ ])$ \\
        $\lambda y.x$                                      & $[x\mapsto 1]$               & $(\circ \ 2 \ [\ ])$                                                     \\
        $\mathtt{cl}(\lambda y. x, [x \mapsto 1]) $        & $[x\mapsto 1]$               & $(\circ \ 2 \ [\ ])$                                                     \\
        $2$                                                & []                           & $(\mathtt{cl}(\lambda y. x, [x \mapsto 1]) \ \circ)$                     \\
        $x$                                                & $[x\mapsto 1].[y \mapsto 2]$ & $\varnothing$                                                            \\
        $1$                                                & $[x\mapsto 1].[y \mapsto 2]$ & $\varnothing$
      \end{tabular}
    \end{table}
  \end{absolutewide}

\end{frame}

\begin{frame}
  \frametitle{Особенности}
  \begin{itemize}
    \item Предназначена для интерпретации $\lambda$-исчисления
    \item \texttt{Call-by-value}
    \item Частичное применение из коробки
    \item Слишком высокоуровневая, чтобы уметь в оптимизацию хвостовой рекурсии
  \end{itemize}
\end{frame}

\section{Выводы}

\begin{frame}{Обобщение}
  \begin{itemize}
    \item[\textbf{SECD}] Простая в реализации, но глуповатая
    \item[\textbf{Кривин}] Поддержка ленивых вычислений, но только \texttt{call-by-name} со всеми вытекающими
    \item[\textbf{ZAM}] Оптимизированная и крутая, используется в \textsc{OCaml}, но довольно сложная
    \item[\textbf{CEK}] Высокоуровневая и простая, но некоторые оптимизации из-за этого реализовать нельзя
  \end{itemize}
\end{frame}

\begin{frame}{Сводная таблица}
  \begin{absolutewide}
    \begin{table}
      \centering
      {\small
        \begin{tabular}{cccc}
                           & \textbf{порядок вычисления} & \textbf{каррирование}   & \textbf{хвостовая рекурсия} \\ \midrule
          \textbf{SECD}    & \texttt{call-by-value}      & {\color{myRed}\faTimes} & нужно докрутить             \\ \midrule
          \textbf{Кривина} & \texttt{call-by-name}       & {\color{teal}\faCheck}  & {\color{myRed}\faTimes}     \\ \midrule
          \textbf{ZAM}     & \texttt{call-by-value}      & {\color{teal}\faCheck}  & {\color{teal}\faCheck}      \\ \midrule
          \textbf{CEK}     & \texttt{call-by-value}      & {\color{teal}\faCheck}  & {\color{myRed}\faTimes}     \\ \bottomrule
        \end{tabular}
      }
    \end{table}
  \end{absolutewide}
\end{frame}


\appendix
\section{}
\begin{frame}[plain, noframenumbering]
  \frametitle{Вопросы}
  \begin{enumerate}
    \item Исполните данный $\lambda$-терм на подходящей машине: $(\lambda x.\lambda y.(y + y))((\lambda x. x x)(\lambda y. y y))(1)$
    \item Какие преимущества каких абстрактных машин объединяет в себе \textbf{ZAM}?
    \item Исполните данный $\lambda$-терм на подходящей машине: $(\lambda x, y, z.(x + y + z)) (1) (2)$
    \item Используется ли машина Кривина в реализации \textsc{Haskell}? Почему?
  \end{enumerate}
\end{frame}
% \section{Источники}
\defbibheading{bibliography}[\bibname]{}

% \begin{frame}[allowframebreaks]
% 	\frametitle{Reading suggestions}
% 	\printbibliography[keyword={suggestion}]
% \end{frame}

\begin{frame}[allowframebreaks]
	\frametitle{Источники}
	\printbibliography
\end{frame}

\end{document}
