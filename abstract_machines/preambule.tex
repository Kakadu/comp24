
\makeatletter
\appto\input@path{{libs/awesome-beamer}, {libs/smile}}
\makeatother

\definecolor{teal}{HTML}{008080}
\usetheme[english, color, coloraccent=teal, listings, footnote, secslide]{awesome}


\usepackage{polyglossia}
\setdefaultlanguage{russian}
\setotherlanguage{english}

\usepackage[
  backend=biber,
  style=alphabetic
]{biblatex}
\usepackage[strict,autostyle]{csquotes}
\nocite{*}
\addbibresource{refs.bib}

\usepackage{fontawesome5}
\def\ergo{\raisebox{.5pt}{\scalebox{.8}{\faCaretRight}}}

\usepackage{contour}
\usepackage{tikzducks}

\usepackage{fontspec}
\setmainfont{CMU Serif}
\setsansfont[
  Ligatures=TeX,
  BoldFont={* Medium},
]{Fira Sans}
\setmonofont[
  Path = ./fonts/,
  Scale = .9,
  Extension = .ttf,
  Contextuals=Alternate,
  BoldFont={*-Bold},
  UprightFont={*-Regular},
]{Fira Code}


\tikzset{bb/.style={draw=tcbcolframe,dash pattern=on 1mm off 1mm,dash phase=0.5mm,tcb@spec,segmentation@style}}
\def\setlinetext#1{\small\color{tcbcolframe}\contourlength{1.5pt}\contour{tcbcolback}{#1}}
% \renewenvironment{block}[1][]{%
% 	\begin{beamerbox}[segmentation code={
% 					\path[bb] (segmentation.west) to node{\setlinetext{#1}} (segmentation.east);
% 				}]{gray}{}%
% 		}{\end{beamerbox}}
\newenvironment{sblock}[1][]{
  \begin{beamerbox}[sidebyside,segmentation code={
          \path[bb] (segmentation.north) to node[rotate=90]{\setlinetext{#1}} (segmentation.south);
        }]{gray}{}%
    }{\end{beamerbox}}

\usepackage{emoji}



% Some label on the background picture
% \addtobeamertemplate{title page}{}{
% 	\tikz[o]\node[anchor=south east,outer sep=0pt] at (current page.south east)
% 	{\fontsize{4}{4}\selectfont\color{white}This image was generated by AI (DALLE 3)};
% }

\usepackage{multicol}

\usepackage{datetime}
\renewcommand{\dateseparator}{.}

\usepackage[verbatim]{lstfiracode}
\lstdefinestyle{firastyleb}{style=FiraCodeStyle,style=smile@lst@base}
\lstdefinestyle{firastylep}{style=FiraCodeStyle,style=smile@lst@plain}
\lstset{
  tabsize=2,
  style=firastylep,
  %apptoliterate={{=> }{{{=> }}}3} % for whatever reason, the space would otherwise be eaten
}
\lstdefinelanguage{myhaskell}{
  language=haskell,
}
\newcommand\hn[2][]{\lstinline[language=myhaskell,#1]{#2}}
\def\inlineblock#1{\tikz[anchor=base,baseline]\node[inner sep=0.3333em,rnd,fill=lightgray!50,anchor=base,baseline=] {#1};}
\newcommand\h[2][]{\inlineblock{\hn[#1]{#2}}}
\makeatletter
\lstnewenvironment{haskell}{\lstset{language=myhaskell}}{}
\lstnewenvironment{xshaskell}{\lstset{language=myhaskell,basicstyle=\smile@lst@style@base\footnotesize}}{}
\lstnewenvironment{xxshaskell}{\lstset{language=myhaskell,basicstyle=\smile@lst@style@base\tiny}}{}
\def\inpuths #1 from #2 to #3;{\lstinputlisting[language=myhaskell,firstline=#2,lastline=#3]{#1}}
\def\inputhsxs #1 from #2 to #3;{\lstinputlisting[language=myhaskell,firstline=#2,lastline=#3,basicstyle=\smile@lst@style@base\footnotesize]{#1}}
\def\inputhsxxs #1 from #2 to #3;{\lstinputlisting[language=myhaskell,firstline=#2,lastline=#3,basicstyle=\smile@lst@style@base\tiny]{#1}}
\makeatother

\usepackage{booktabs}
% \usepackage{arydshln}
% \usepackage[utf8]{inputenc}
\usepackage{array}
\usepackage{colortbl}
\usepackage{xcolor}
\usepackage{caption}
\usepackage{amsmath}
\usepackage{mathtools}
\usepackage{emoji}
\usepackage{appendix}

\definecolor{myPurple}{HTML}{4D0694}
\definecolor{myRed}{HTML}{940606}