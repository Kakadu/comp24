% \tikzset{
% 	u/.style={roundednode,draw=none,shadow,fill=gray!20},
% 	% u/.style={roundednode,fill=gray!20},
% 	v/.style={arrows={-Stealth[round,inset=0pt,length=10pt,angle'=90,open]},lw,lcr,rnd},
% }
% \def\card#1#2{\begin{tikzpicture}[node distance=0]
% 	\node[text width=0.45\textwidth] (C) {#2};
% 	\node[above=of C,yshift=-2mm] (T) {#1};
% 	\node[u,fit=(T)(C),node on layer=background] () {};
% \end{tikzpicture}}
% \begin{frame}
% 	\frametitle{Summary}
% 	\begin{wide}
% 		\begin{tikzpicture}
% 			\node (L) {\card{Lens}{\begin{itemize}
% 				\item Focus on a single part of a data structure
% 				\item \hn{^.} returns the focused part directly
% 			\end{itemize}}};
% 			\node[right=of L] (P) {\card{Prism}{\begin{itemize}
% 				\item Focus on a single part that may not exist
% 				\item \hn{^?} returns the focused part inside a \hn{Maybe}
% 			\end{itemize}}};
% 			\coordinate (A) at ($(L)!.5!(P)$);
% 			\coordinate (M) at (A |- L.north);
% 			\node[above=of M] (T) {\card{Traversals}{\begin{itemize}
% 				\item Focus on multiple parts (also zero) of a data structure
% 				\item \hn{^..} returns list of the focused parts
% 			\end{itemize}}};
% 			\draw[v] (L) to (T);
% 			\draw[v] (P) to (T);
% 		\end{tikzpicture}
% 	\end{wide}
% \end{frame}