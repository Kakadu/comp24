% !TEX TS-program = pdflatex
% !TeX spellcheck = ru_RU
% !TEX root = demo.tex
\documentclass[aspectratio=169
  %, draft      % speedup compilation
  %, handout    % eliminate pauses
  , xcolor={svgnames}
  , russian  % This line affects translation of theorem titles
  ]{beamer}

%%%%%%%%%%%%%%%%%%%%%%%%%%%%%%%%%%%%%%%%
\input{../preamble}
\bibliography{../bibliography.bib}
%%%%%%%%%%%%%%%%%%%%%%%%%%%%%%%%%%%%%%%%
\title[Компиляторы. ANF]{Компиляторы.  ANF}
\date{\DTMDate{2024-09-28}}


\AtBeginSection[]
{
  \begin{frame}<beamer>
    \frametitle{Оглавление}
    \tableofcontents[currentsection,currentsubsection]
  \end{frame}
}

\begin{document}
\maketitle

\defverbatim[colored]{\ANFInput}{
\begin{lstlisting}[language=ocaml]
type expr =
| Num of int
| Id of string
| Plus of expr * expr
| Minus of expr * expr
| Let of string * expr * expr
\end{lstlisting}
}

\defverbatim[colored]{\ANFI}{
\begin{lstlisting}[language=ocaml]
type immexpr =
| ImmNum of int
| ImmId of string

type cexpr =
| CPlus of immexpr * immexpr

type aexpr =
| ALet of string * cexpr * aexpr
\end{lstlisting}
}

\begin{frame}{ANF ---~упрощенное представление программ}
Мы будем упрощать выражения, чтобы аргументами функций были \enquote{атомарные} (immediate) выражения (константы, имена и т.д.)

\begin{minipage}{0.35\linewidth}
\begin{minipage}{3cm}\ANFInput\end{minipage}
\end{minipage}$\Huge \quad \Longrightarrow \quad $
\begin{minipage}{0.35\linewidth}
\begin{minipage}{3cm}\ANFI\end{minipage}
\end{minipage}
\footnotetext{Впервые упомянуты в 1993 в статье \foreignquote{english}{The Essence of Compiling with Continuations}~\cite{Flanagan1993} }
\end{frame}

\defverbatim[colored]{\ANFII}{
\begin{lstlisting}[escapechar=!, language=ocaml]
type cexpr =
| CPlus of immexpr * immexpr
| !\tikzmark{s1}!CImmExpr of immexpr!\tikzmark{e1}!

type aexpr =
| ALet of string * cexpr * aexpr
| !\tikzmark{s2}!ACExpr of cexpr!\tikzmark{e2}!
\end{lstlisting}
}


\begin{frame}{ANF: Примеры и уточнения}
\begin{columns}[T]
\begin{column}[t]{.49\textwidth}
\begin{enumerate}
\item \lstinline|(5 + 4) - 2|\\
  \uncover<2->{ \lstinline|let v1 = 5 + 4 in v1 - 2| }
\item \lstinline|let x = 5 in x|
\end{enumerate}
\end{column}
\begin{column}[t]{.49\textwidth}
  \begin{minipage}{3cm}\ANFI\end{minipage}
\end{column}
\end{columns}
\end{frame}

\begin{frame}{ANF представление --- новые конструкторы}
  \begin{minipage}{3cm}\ANFII\end{minipage}

  \begin{tikzpicture}[use tikzmark]
      \tikzHighlight(s1)(e1);
      \tikzHighlight(s2)(e2);
  \end{tikzpicture}
\end{frame}

\defverbatim[colored]{\ANFImplI}{
\begin{lstlisting}[language=ocaml]
let rec anf (e : expr) (expr_with_hole : immexpr -> aexpr) = function
  | Num n -> expr_with_hole (ImmNum n)
  | Id x -> expr_with_hole (ImmId x)
  (* | ... *)
\end{lstlisting}
}
\defverbatim[colored]{\ANFImplII}{
\begin{lstlisting}[language=ocaml]
  | Plus(left, right) ->
      anf left (fun limm ->
        anf right (fun rimm ->
\end{lstlisting}
}
\defverbatim[colored]{\ANFImplIII}{
\begin{lstlisting}[language=ocaml]
          (* bad *) CPlus (limm, rimm)
\end{lstlisting}
}
\defverbatim[colored]{\ANFImplIV}{
\begin{lstlisting}[language=ocaml]
          (* OK *) ALet("result_of_plus",
                     CPlus(limm, rimm),
                     expr_with_hole (ImmId "result_of_plus"))
      ))
\end{lstlisting}
}
\begin{frame}{ANF представление --- сложный случай}
\lstinline|Plus(..., ...)|\\

Не совсем понятно, что делать с результатами левого и правого слагаемого, так как там могут появиться много \lstinline=let=.
Мэтчиться по ним не хочется. Поэтому...

\ANFImplI \pause
\vspace{-1em}
\ANFImplII
\vspace{-1em}
\uncover<2> { \ANFImplIII }
\uncover<3> { \ANFImplIV }
\end{frame}

%\newcommand{\xarr}[1]{\ensuremath{\xrightarrow{#1}}}
\newcommand{\pure}{\ensuremath{\xrightarrow{pure}}}

\defverbatim[colored]{\mlPlusSign}{
\begin{lstlisting}[escapeinside={!}{!}, language=ocaml]
val (+): int !\pure! int !\pure! int
\end{lstlisting}
}

\begin{frame}{Заключение про ANF}
\begin{itemize}
\item ANF пригодится, когда дойдете до LLVM
\item Можно написать много разных реализаций, самое нетривиальные места --- \lstinline|let| и \lstinline|if|

\item Часто будут появляться ненужные переменные, например, \lstinline|let x1 = x3 in x1|. Можно пытаться такое не порождать, а можно упрощать отдельным проходом

\item В контексте miniML могут появляться каверзные случаи, типа  \lstinline|let temp1 = (+) x1 in temp1 x2| вместо \lstinline|x1 + x2|. \\
Используя информацию об уже объявленных функциях, нужно не создавать такие ANF
\begin{minipage}{.5\textwidth}
\mlPlusSign
\end{minipage}
\item Вывод: если сделать проверку типов до ANF, то будут менее \enquote{мусорные} ANF
\end{itemize}

\end{frame}


\begin{frame}{Просто картинка}
\framesubtitle{}
\begin{columns}[T]
\begin{column}{.65\textwidth}

Не пихайте все tikz в один файл. Лучше создать отдельный PDF, либо tikzexternalize

  \begin{definition}[Граф потока управления]
  $\langle V, E, Entry\rangle$, где
  \begin{itemize}
  \item $V$ вершины для представления инструкций или базовых блоков
  \item $E$ рёбра потенциального потока управления, $E\subseteq V\times V$
  \item $Entry\in V$ --- уникальная точка входа
  \end{itemize}
  \end{definition}

\end{column}\hspace{1em}
\begin{column}{.3\textwidth}
  \vspace{-1em}
  \includegraphics[scale=1.2]{../misc/cfg1.pdf}
\end{column}
\end{columns}
\end{frame}

\begin{frame}%[allowframebreaks]
\frametitle<presentation>{Ссылки \& Acknowledgements}
\phantom{\cite*{PolitzANF2016, Flanagan1993}}
\vspace{-1em}
\printbibliography
\end{frame}


\end{document}
