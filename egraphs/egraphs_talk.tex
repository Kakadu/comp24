% !TEX TS-program = xelatex
% !TeX spellcheck = ru_RU
% !TEX root = egraphs_talk.tex
% !BIB program = bibtex

\documentclass[aspectratio=169
  %, draft      % speedup compilation
  %, handout    % eliminate pauses
  , xcolor={svgnames}
  , russian  % This line affects translation of theorem titles
  ]{beamer}

%%%%%%%%%%%%%%%%%%%%%%%%%%%%%%%%%%%%%%%%
\input{../preamble}
\bibliography{egraphs_talk.bib}
%%%%%%%%%%%%%%%%%%%%%%%%%%%%%%%%%%%%%%%%
\title[E-Graphs]{E-Graphs. Их применение в компиляторах и не только}
\date{\DTMDate{2024-09-28}}

\definecolor{ForestGreen} {HTML}{009B55}
\AtBeginSection[]
{
  \begin{frame}<beamer>
    \frametitle{Оглавление}
    \tableofcontents[currentsection,currentsubsection]
  \end{frame}
}

\usepackage{amsmath}
\usepackage{tikz}
\usepackage{xcolor}
\usepackage{multicol}
\usepackage{graphicx}
\usepackage{animate}
% in preamble

% in documenet
\usepackage{xmpmulti}
\begin{document}
\maketitle


\begin{frame}{Переписывание выражений}

    \only<1>
{
\centering

% Main expression in the middle
\Huge
\text{$(a\; *\; 2)\; /\; 2\rightarrow$ $a$}

\vspace{1cm}


}

    \only<2>
{
\centering

% Main expression in the middle
\Huge
\text{$(a\; *\; 2)\; /\; 2\rightarrow$ $a$}

\vspace{1cm}

% "Rewrite it" below the main expression
\Large
\text{Как переписать?}

\vfill

}

    \only<3>
{
\centering

% Main expression in the middle
\Huge
\text{$(a\; *\; 2)\; /\; 2\rightarrow$ $a$}

\vspace{1cm}

% "Rewrite it" below the main expression
\Large
\text{Как переписать?}

\vfill

% Columns for b1 and b2
\begin{multicols}{2}
    \textcolor{ForestGreen}{Полезные правила переписывания}
    \begin{itemize}
        \item $(x\; * \;y) \;/ \;z \;= \;x\; * \;(y \;/ \;z)$
        \item $x \; / \; x \; = \; 1$
        \item $x \; * \; 1 \; = \; x$
    \end{itemize}

    \columnbreak

    {\fontsize{13.1}{12}\selectfont \textcolor{red}{Бесполезные правила переписывания}}
    \begin{itemize}
        \item $x\; * \;2 \; = \;  x \; << \; 1$
        \item $x \;*\; y \;= \;y \;* \;x$
        \item $x \;= \;x \;* \;1$
    \end{itemize}
\end{multicols}
}
\end{frame}

\begin{frame}{Переписывание выражений. Happy path}

    \only<1>
{
    \Huge{ \centering
    $(a \; *\; 2)\; / \;2$
    }

    \vspace{1cm} % Adds some space between the expression and the bullet list

    {\fontsize{16.1}{12}\selectfont \textcolor{green}{Полезные правила переписывания}}
    {\fontsize{16.1}{12}\selectfont % Adjust the font size here
    \begin{itemize}
        \item $(x\; * \;y) \;/ \;z \;= \;x\; * \;(y \;/ \;z)$
        \item $x \; / \; x \; = \; 1$
        \item $x \; * \; 1 \; = \; x$
    \end{itemize}
    }
}

    \only<2>
{
    \Huge{ \centering
    $(a \; *\; 2)\; / \;2 \rightarrow a \;*\; (2\; /\; 2)$
    }

    \vspace{1cm} % Adds some space between the expression and the bullet list

    {\fontsize{16.1}{12}\selectfont \textcolor{ForestGreen}{Полезные правила переписывания}}
    {\fontsize{16.1}{12}\selectfont % Adjust the font size here
    \begin{itemize}
        \item $(x\; * \;y) \;/ \;z \;= \;x\; * \;(y \;/ \;z)$
        \item $x \; / \; x \; = \; 1$
        \item $x \; * \; 1 \; = \; x$
    \end{itemize}
    }
}

    \only<3>
{
    \Huge{ \centering
    $(a \; *\; 2)\; / \;2 \rightarrow a \;*\; (2\; /\; 2) \rightarrow a \;*\; 1$
    }

    \vspace{1cm} % Adds some space between the expression and the bullet list

    {\fontsize{16.1}{12}\selectfont \textcolor{ForestGreen}{Полезные правила переписывания}}
    {\fontsize{16.1}{12}\selectfont % Adjust the font size here
    \begin{itemize}
        \item $(x\; * \;y) \;/ \;z \;= \;x\; * \;(y \;/ \;z)$
        \item $x \; / \; x \; = \; 1$
        \item $x \; * \; 1 \; = \; x$
    \end{itemize}
    }
}

    \only<4>
{
    \Huge{ \centering
    $(a \; *\; 2)\; / \;2 \rightarrow a \;*\; (2\; /\; 2) \rightarrow a \;*\; 1 \rightarrow a$
    }

    \vspace{1cm} % Adds some space between the expression and the bullet list

    {\fontsize{16.1}{12}\selectfont \textcolor{ForestGreen}{Полезные правила переписывания}}
    {\fontsize{16.1}{12}\selectfont % Adjust the font size here
    \begin{itemize}
        \item $(x\; * \;y) \;/ \;z \;= \;x\; * \;(y \;/ \;z)$
        \item $x \; / \; x \; = \; 1$
        \item $x \; * \; 1 \; = \; x$
    \end{itemize}
    }
}
\end{frame}

\begin{frame}{Переписывание выражений. Проблемные случаи}

    \only<1>
{
    \LARGE{ \centering
    $(a \; *\; 2)\; / \;2 \rightarrow (a \;<<\; 1) \;/ \;2 \rightarrow \textcolor{red}{\Huge \times}$
    } \newline \newline

    \vspace{1cm} % Adds some space between the expression and the bullet list

    {\fontsize{15.1}{12}\selectfont \textcolor{red}{Бесполезные правила переписывания}}
    {\fontsize{16.1}{12}\selectfont % Adjust the font size here
    \begin{itemize}
        \item $x\; * \;2 \; = \;  x \; << \; 1$
        \item $x \;*\; y \;= \;y \;* \;x$
        \item $x \;= \;x \;* \;1$
    \end{itemize}
    }
}

\only<2>
{
    \LARGE{ \centering
    $(a \; *\; 2)\; / \;2 \rightarrow (a \;<<\; 1) \;/ \;2 \rightarrow \textcolor{red}{\Huge \times}$
    } \newline \newline
    \LARGE{ \centering
    $\textcolor{blue}{(a \;*\; 2) \;/\; 2} \rightarrow (2 \;*\; a) \;/\; 2 \rightarrow \textcolor{blue}{(a \;*\; 2) \;/\; 2}$
    }

    \vspace{1cm} % Adds some space between the expression and the bullet list

    {\fontsize{15.1}{12}\selectfont \textcolor{red}{Бесполезные правила переписывания}}
    {\fontsize{16.1}{12}\selectfont % Adjust the font size here
    \begin{itemize}
        \item $x\; * \;2 \; = \;  x \; << \; 1$
        \item $x \;*\; y \;= \;y \;* \;x$
        \item $x \;= \;x \;* \;1$
    \end{itemize}
    }
}

\only<3>
{
    \LARGE{ \centering
    $(a \; *\; 2)\; / \;2 \rightarrow (a \;<<\; 1) \;/ \;2 \rightarrow \textcolor{red}{\Huge \times}$
    } \newline \newline
    \LARGE{ \centering
    $\textcolor{blue}{(a \;*\; 2) \;/\; 2} \rightarrow (2 \;*\; a) \;/\; 2 \rightarrow \textcolor{blue}{(a \;*\; 2) \;/\; 2}$
    } \newline \newline
    \LARGE{ \centering
    $a \rightarrow a \; * \; 1 \rightarrow a \; * \; 1 \;*\; 1 \rightarrow .\;.\;.$
    }

    \vspace{1cm} % Adds some space between the expression and the bullet list

    {\fontsize{15.1}{12}\selectfont \textcolor{red}{Бесполезные правила переписывания}}
    {\fontsize{16.1}{12}\selectfont % Adjust the font size here
    \begin{itemize}
    \item $x\; * \;2 \; = \;  x \; << \; 1$
    \item $x \;*\; y \;= \;y \;* \;x$
    \item $x \;= \;x \;* \;1$
    \end{itemize}
    }
}
\end{frame}

\begin{frame}{Решение проблемы 1/3}
  \only<1>
    {
        \centering
        \includegraphics[width=13cm, height=5cm]{misc/egraphs_images/naive/n-0.jpg}
   }
   \only<2>
    {
        \centering
        \includegraphics[width=13cm, height=5cm]{misc/egraphs_images//naive/n-1.jpg}
   }
   \only<3>
    {
        \centering
        \includegraphics[width=13cm, height=5cm]{misc/egraphs_images/naive/n-2.jpg}
   }
   \only<4>
    {
        \centering
        \includegraphics[width=13cm, height=5cm]{misc/egraphs_images/naive/n-3.jpg}
   }
    \only<5>
    {
        \centering
        \includegraphics[width=13cm, height=5cm]{misc/egraphs_images/naive/n-4.jpg}
   }
    \only<6>
    {
        \centering
        \includegraphics[width=13cm, height=5cm]{misc/egraphs_images/naive/n-5.jpg}
   }
   \only<7>
    {
        \centering
        \includegraphics[width=13cm, height=5cm]{misc/egraphs_images/naive/n-6.jpg}
   }
   \only<8>
    {
        \centering
        \includegraphics[width=13cm, height=5cm]{misc/egraphs_images/naive/n-7.jpg}
   }
   \only<9>
    {
        \centering
        \includegraphics[width=13cm, height=5cm]{misc/egraphs_images/naive/n-8.jpg}
   }
   \only<10>
    {
        \centering
        \includegraphics[width=13cm, height=5cm]{misc/egraphs_images/naive/n-9.jpg}
   }
   \only<11>
    {
        \centering
        \includegraphics[width=13cm, height=5cm]{misc/egraphs_images/naive/n-10.jpg}
   }
   \only<12>
    {
        \centering
        \includegraphics[width=13cm, height=5cm]{misc/egraphs_images/naive/n-11.jpg}
   }
   \only<13>
    {
        \centering
        \includegraphics[width=13cm, height=5cm]{misc/egraphs_images/naive/n-12.jpg}
   }
   \only<14>
    {
        \centering
        \includegraphics[width=13cm, height=5cm]{misc/egraphs_images/naive/n-13.jpg}
   }
   \only<15>
    {
        \centering
        \includegraphics[width=13cm, height=5cm]{misc/egraphs_images/naive/n-14.jpg}
   }
   \only<16>
    {
        \centering
        \includegraphics[width=13cm, height=5cm]{misc/egraphs_images/naive/n-15.jpg}
   }
   \only<17>
    {
        \centering
        \includegraphics[width=13cm, height=5cm]{misc/egraphs_images/naive/n-16.jpg}
   }
\end{frame}

\begin{frame}{Решение проблемы 2/3}
    \begin{center}
            \includegraphics[width=3.2cm, height=4.2cm]{misc/egraphs_images/1_11new.png}
            \hspace{0.5cm}
            \raisebox{11.5ex}{$\xrightarrow{\text{\fontsize{10.0}{12} $a = b$}}$}
            \hspace{0.5cm}
            \includegraphics[width=5.2cm, height=4.2cm]{misc/egraphs_images/1_22new.jpg}
    \end{center}
\end{frame}

\begin{frame}{Решение проблемы 2/3}
    \centering
        \includegraphics[width=8cm, height=5cm]{misc/egraphs_images/shared_im.jpg}
\end{frame}

\begin{frame}{Решение проблемы 3/3}
    \centering
        \includegraphics[width=5cm, height=6cm]{misc/egraphs_images/eclass_new.jpg}
\end{frame}





\begin{frame}{E-Graphs}
    \begin{block}{Определение}
        E-Graph --- это структура данных, предназначенная для хранения отношений эквивалентности над выражениями некоторого языка.
    \end{block}

    \begin{columns}
        \begin{column}{0.5\textwidth}
            \centering
            \includegraphics[width=5cm, height=5cm]{misc/egraphs_images/egraph_1.png} % Left middle image
        \end{column}
        \begin{column}{0.5\textwidth}
            \centering
            \includegraphics[width=5.2cm, height=4.2cm]{misc/egraphs_images/enodes_eclasses.jpg} % Right middle image
        \end{column}
    \end{columns}

\end{frame}

\begin{frame}{E-Graph. Насыщение 1/3}
    \only<1>
    {
        \begin{center}
            \includegraphics[width=5.2cm, height=4.2cm]{misc/egraphs_images/egraph_1.png}
        \end{center}
    }
    \only<2>
    {
        \begin{center}
            \includegraphics[width=5.2cm, height=4.2cm]{misc/egraphs_images/egraph_1.png}
            \hspace{0.5cm}
            \raisebox{11.5ex}{$\xrightarrow{\text{\fontsize{10.0}{12} $x\; * \;2 \; = \; x \; << \; 1$}}$}
            \hspace{0.5cm}
            \includegraphics[width=5.2cm, height=4.2cm]{misc/egraphs_images/egraph_2.jpg}
        \end{center}
    }
\end{frame}

\begin{frame}{E-Graph. Насыщение 2/3}
    \only<1>
    {
        \begin{center}
            \includegraphics[width=5.2cm, height=4.2cm]{misc/egraphs_images/egraph_2.jpg}
        \end{center}
    }
    \only<2>
    {
        \begin{center}
            \includegraphics[width=5.2cm, height=4.2cm]{misc/egraphs_images/egraph_2.jpg}
            \hspace{0.5cm}
            \raisebox{11.5ex}{$\xrightarrow{\text{\fontsize{9.4}{12} $(x*y)/z = x*(y/z)$}}$}
            \hspace{0.5cm}
            \includegraphics[width=5.2cm, height=4.2cm]{misc/egraphs_images/egraph_3.jpg}
        \end{center}
    }
\end{frame}

\begin{frame}{E-Graph. Насыщение 3/3}
    \only<1>
    {
        \begin{center}
            \includegraphics[width=5.2cm, height=4.2cm]{misc/egraphs_images/egraph_3.jpg}
        \end{center}
    }
    \only<2>
    {
        \begin{center}
            \includegraphics[width=5.2cm, height=4.2cm]{misc/egraphs_images/egraph_3.jpg}
            \hspace{0.5cm}
            \raisebox{11.5ex}
            {$\xrightarrow[\text{\fontsize{10.4}{12} $x / x = 1$}]{\text{\fontsize{10.4}{12} $x * 1 = x$}}$}
            \hspace{0.5cm}
            \includegraphics[width=5.2cm, height=4.2cm]{misc/egraphs_images/egraphs4.jpg}

        \end{center}

        \begin{center}
           \textbf{Так же содержит циклы, $a * 1$, $a * 1 * 1$ ...} \\
        \end{center}

    }
\end{frame}

\begin{frame}{Equality saturation}
    \begin{center}
        \includegraphics[width=15.2cm, height=5.2cm]{misc/egraphs_images/equality_saturation1.jpeg}
    \end{center}

    \begin{center}
       \begin{itemize}
           \item \textbf{Инвариант: если $a = b$, то $f(a) = f(b)$} \\
           \item \textbf{Извлечение} --- NP-complete проблема \\
       \end{itemize}


    \end{center}
\end{frame}

\begin{frame}{E-graphs Community}
    \begin{itemize}
        \item \textbf{egg} (\textbf{eg}raphs \textbf{g}ood)\textit{ $-$ A flexible, high-performance e-graph library} (Rust)\newline
        \item \textbf{Ego} (\textbf{Eg}raphs in \textbf{O}Caml) $-$ Библиотека для работы с е-графами и equality saturation в общих чертах повторяющая дизайн egg (OCaml)\newline

        \textbf{demo}

    \end{itemize}
\end{frame}

\begin{frame}{}
    \centering
    \includegraphics[width=10.2cm, height=7.2cm]{misc/egraphs_images/demo_egraph.jpeg}
\end{frame}

\begin{frame}{}
    \centering
    \textbf{\Huge Примеры оптимизаций, выразимых с E-Graphs} \\
    \vspace{1cm}
\end{frame}

\begin{frame}{Algebraic Simplifications}
    \begin{itemize}
    \item \textbf{\fontsize{14.1}{12} Нейтральный элемент по сложению:} \textbf{\fontsize{14.1}{12} $a + 0 = a$}
    \item \textbf{\fontsize{14.1}{12} $0 + a = a$}
    \newline
    \item \textbf{\fontsize{14.1}{12} Нейтральный элемент по умножению:} \textbf{\fontsize{14.1}{12} $a \cdot 1 = a$}
    \item \textbf{\fontsize{14.1}{12} $1 \cdot a = a$}
    \newline
    \item \textbf{\fontsize{14.1}{12} Обратный элемент по умножению:} \textbf{\fontsize{14.1}{12} $a \cdot 0 = 0$}
    \item \textbf{\fontsize{14.1}{12} $0 \cdot a = 0$}
    \newline
    \item \textbf{\fontsize{14.1}{12} Нейтральный элемент по делению:} \textbf{\fontsize{14.1}{12} $a \; / \; 1 = a$}
    \newline
    \item \textbf{\fontsize{14.1}{12} ...}
\end{itemize}
\end{frame}

\begin{frame}{}
    \centering
    \includegraphics[width=10.2cm, height=7.2cm]{misc/egraphs_images/algebraic_simpl_demo.jpeg}
\end{frame}

\begin{frame}{Strength reduction}
    \begin{itemize}
    \item \textbf{\fontsize{14.1}{12} Умножение на степень двойки:} \textbf{\fontsize{14.1}{12} $a \; * \; 2^n = a << n$}
    \newline
    \item \textbf{\fontsize{14.1}{12} Деление на степень двойки:} \textbf{\fontsize{14.1}{12} $a \; / \;  2^n = a >> n$}
    \newline
    \item \textbf{\fontsize{14.1}{12} Проверка четности:} \textbf{\fontsize{14.1}{12} $x \; \% \; 2 = x \; \& \; 1$}
    \newline
    \item \textbf{\fontsize{14.1}{12} ...}
\end{itemize}
\end{frame}

\begin{frame}{}
    \centering
    \includegraphics[width=10.2cm, height=7.2cm]{misc/egraphs_images/strength_red.jpeg}
\end{frame}

\begin{frame}{Function inlining}
    \begin{itemize}
    \item \textbf{\fontsize{14.1}{12} Замена вызовов функции непосредственно её телом}

\end{itemize}
\end{frame}

\begin{frame}{}
    \centering
    \includegraphics[width=7.2cm, height=7.2cm]{misc/egraphs_images/inlining_demo.jpeg}
\end{frame}

\begin{frame}{Common subexpression elimination}
    \centering
    \includegraphics[width=12.2cm, height=5.2cm]{misc/egraphs_images/cse.jpeg}
\end{frame}

\begin{frame}{}
    \centering
    \includegraphics[width=4.2cm, height=6.2cm]{misc/egraphs_images/cse_demo.jpeg}
\end{frame}

\begin{frame}{Прочие оптимизации}
    \begin{itemize}
    \item \textbf{\fontsize{14.1}{12} Constant folding \& Constant propagation }
    \newline
    \item \textbf{\fontsize{14.1}{12} CSE for Arrays (Удаление избыточного доступа к массиву) }
    \newline
    \item \textbf{\fontsize{14.1}{12}  Loop Peeling (Вытаскиваем первую итерацию цикла)}
    \newline
    \item \textbf{\fontsize{14.1}{12}  Tail-call elimination}
    \newline
    \item \textbf{\fontsize{14.1}{12}  ...}
    \newline
    \end{itemize}
\end{frame}

\begin{frame}{Equality saturation \& E-Graphs. Pros and cons}
    \begin{columns}
        \column{0.5\textwidth}
        \textbf{Преимущества}
        \begin{itemize}
            \item \textbf{Выразимость}
            \item \textbf{Инкрементальное обновление}
            \item \textbf{Оптимальность}
        \end{itemize}

        \column{0.5\textwidth}
        \textbf{Недостатки}
        \begin{itemize}
            \item \textbf{Memory usage}
            \item \textbf{Perfomance overhead (слияние нодов)}
            \item \textbf{NP-полнота извлечения}
        \end{itemize}
    \end{columns}
\end{frame}

\begin{frame}{Где ещё можно встретить e-graphs?}
    \begin{itemize}
    \item \textbf{\fontsize{14.1}{12} Theorem provers (Lean) \cite{Lean}}
    \newline
    \item \textbf{\fontsize{14.1}{12} Deep learning (Graph rewriting)\cite{Deep_learning}}
    \newline
    \item \textbf{\fontsize{14.1}{12} SMT-solvers\cite{CMT} }
    \newline
    \item \textbf{\fontsize{14.1}{12} ... }
    \end{itemize}
\end{frame}

\begin{frame}{Вопросы}
    \begin{itemize}
    \item \textbf{\fontsize{14.1}{12} 1. Переписывание выражений. Правила переписывания. Полезные и бесполезные правила. Примеры. }
    \newline
    \item \textbf{\fontsize{14.1}{12}
2. E-графы. Equality saturation. Примеры оптимизаций выразимых с помощью equality saturation и е-графов.}
    \newline
    \item \textbf{\fontsize{14.1}{12} 3. Изобразить е-граф для выражения $(a + 0) + (a * 1)$ и применить к нему следующие правила: $a + 0 = a$, $a * 1 = a$ }
    \newline
    \item \textbf{\fontsize{14.1}{12} 4. Преимущества и недостатки equality saturation и e-графов
 }
    \end{itemize}
\end{frame}

\begin{frame}%[allowframebreaks]
\frametitle<presentation>{Ссылки \& Acknowledgements}
\phantom{\cite*{Deep_learning,CMT,Lean}}
\vspace{-1em}
\printbibliography
\end{frame}


\end{document}
