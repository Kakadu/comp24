% color options
\definecolor{YellowGreen} {HTML}{B5C28C}
\definecolor{ForestGreen} {HTML}{009B55}
\definecolor{MyPurple}{HTML}{7f007f}
\definecolor{CommentColor}{HTML}{3f7f5f}
\def\HaskellTypeclassColor\PYG{k+kt}
\def\HaskellCommentColor\PYG{c+c1}

\newif\ifminted
\newif\iflistings

\listingstrue
%\mintedtrue

\iflistings
    \usepackage{listings}
    %\input{../ss24/riscv-asm.tex}
    %\newcommand{\inline}[1]{\lstinline{haskell}{#1}}
    % TODO: https://tex.stackexchange.com/questions/4198/emphasize-word-beginning-with-uppercase-letters-in-code-with-lstlisting-package
    \definecolor{eclipseGreen}{RGB}{63,127,95}

    % https://tex.stackexchange.com/a/4199
    \makeatletter
    \newcommand*\ocamlidstyle{%
            \expandafter\id@style\the\lst@token\relax
    }
    \def\id@style#1#2\relax{%
            \ifcat#1\relax\else
                    \ifnum`#1=\uccode`#1%
                             \ttfamily\bfseries\color{MyPurple}
                    \else
                                                \ttfamily
                    \fi
            \fi
    }
    \makeatother

    \lstdefinelanguage{none}{ identifierstyle= }
    \lstdefinelanguage{menhir}
      { identifierstyle=
      , morecomment=[s]{/*}{*/}
      , commentstyle=\color{eclipseGreen} % style of comments
      , classoffset=0
      , keywords={ left, nonassoc, start,  token, type
      }
      , keywordstyle=\ttfamily\bfseries\color{MyPurple}
    }
    \lstdefinelanguage{ocamllex}
      { identifierstyle=
      , morecomment=[s]{/*}{*/}
      , commentstyle=\color{eclipseGreen} % style of comments
      , classoffset=0
      , keywords={ and, rule, parsem if, then, else, eof, as, raise, parse
      }
      , keywordstyle=\ttfamily\bfseries\color{MyPurple}
      , stringstyle=\color{blue}
      , morestring=*[d]{"}
      , morestring=*[d]{'}
    }
    \ifxetexorluatex
    \lstdefinelanguage{ocaml}{
        basicstyle=\ttfamily   % Вот тут надо стиль ставить, а не у идентификаторов
        %, identifierstyle=\ocamlidstyle
        , identifierstyle=\ttfamily
        %, commentstyle=\HaskellCommentColor\itshape\HaskellCommentColor
        , sensitive=true
        %
        , classoffset=0
        , keywords={ fun, function, and, let, rec, in, match, with, when
            , class, type, of, do, done, as, val
            , inherit, module, struct, sig, include
            , if, then, else, while
            , try, exception, raise
            , mod
            , assert, true, false, begin, end, lazy
            , @@deriving
            , Some, None
        }
        , keywordstyle=\ttfamily\bfseries\color{MyPurple} %\underbar
        , classoffset=1
        , morekeywords={pure,empty,select,branch,oneOf}
        , keywordstyle=\color{MyPurple}
        , classoffset=2
        , morekeywords={Monad,Applicative,Selective,String
            ,Either,Left,Right
            ,Maybe,Some,None
        }
        , keywordstyle=\ttfamily\bfseries\color{MyPurple}
        , classoffset=0,
        %keywordstyle=[2]{\color{orange}},
        otherkeywords={::},
        %identifierstyle=\fontfamily{cmtt}\selectfont\ttfamily,
        %basewidth={0.5em,0.5em},
        columns=fixed,
        %fontadjust=true,
        %literate={->}{{$\to$}}3 {===}{{$\equiv$}}1 {=/=}{{$\not\equiv$}}1 {|>}{{$\triangleright$}}3 {\\/}{{$\vee$}}2 {/\\}{{$\wedge$}}2 {>=}{{$\ge$}}1 {<=}{{$\le$}} 1,
        , morecomment=[s]{(*}{*)}
        , commentstyle=\color{eclipseGreen} % style of comments
        %, literate={\$}{{\textcolor{blue}{\$}}}1
        %, literate={<\$>}{{\textcolor{RawSienna}{\ <\$>\ } }}1
        %           {>?>}{{\textcolor{RawSienna}{\ >?>\ } }}1
        , string = [d]{"}
        , showstringspaces=false
        , stringstyle = \color{black}
        %, morestring = [d][\color{orange}]{'}
        %, moredelim={[s][\color{orange}\ttfamily]{'}{'}}
    }
    \else
        \lstdefinelanguage{ocaml}{
            % modified minial style for pdflatex
            basicstyle=\ttfamily
            , identifierstyle=\ttfamily
            , sensitive=true
            , classoffset=0
            , extendedchars=true
            , literate=
                {а}{{a}}1 {б}{{b}}1 {в}{{v}}1 {г}{{g}}1 {д}{{d}}1
                {е}{{e}}1 {ё}{{e}}1 {ж}{{zh}}1 {з}{{z}}1 {и}{{i}}1 {й}{{j}}1
                       {к}{{k}}1 {л}{{l}}1 {м}{{m}}1 {н}{{n}}1 {о}{{o}}1
                       {п}{{p}}1 {р}{{r}}1 {с}{{s}}1 {т}{{t}}1 {у}{{u}}1
                       {ф}{{f}}1 {х}{{h}}1 {ц}{{ts}}1 {ч}{{ch}}1 {ш}{{sh}}1
                       {щ}{{sch}}1 {ъ}{{}}1 {ы}{{y}}1 {ь}{{}}1 {э}{{e}}1
                       {ю}{{u}}1  {я}{{a}}1
                {А}{{a}}1 {Б}{{b}}1 {В}{{v}}1 {Г}{{g}}1 {Д}{{d}}1
                {Е}{{e}}1 {Ё}{{e}}1 {Ж}{{zh}}1 {З}{{z}}1 {И}{{i}}1
                {Й}{{j}}1 {К}{{k}}1 {Л}{{l}}1 {М}{{m}}1 {Н}{{n}}1
                {О}{{o}}1 {П}{{p}}1 {Р}{{r}}1 {С}{{s}}1 {Т}{{t}}1
                {У}{{u}}1 {Ф}{{f}}1 {х}{{h}}1 {ц}{{ts}}1 {Ч}{{ch}}1
                {Ш}{{sh}}1 {Щ}{{sch}}1 {Ъ}{{}}1 {Ы}{{y}}1 {Ь}{{}}1
                {Э}{{e}}1 {Ю}{{u}}1  {Я}{{a}}1
            , keywords={ fun, function, and, let, rec, in, match, with, when
                , class, type, of, do, done, as, val
                , inherit, module, struct, sig, include
                , if, then, else, while
                , try, exception, raise
                , mod
                , assert, true, false, begin, end, lazy
                , @@deriving
                , Some, None
            }
            , keywordstyle=\ttfamily\bfseries\color{MyPurple} %\underbar
            , morecomment=[s]{(*}{*)}
            , commentstyle=\color{eclipseGreen} % style of comments
            , string = [d]{"}
            , showstringspaces=false
            , stringstyle = \color{black}
            %, moredelim={[s][\color{eclipseGreen}\ttfamily]{'}{'}}
        }
    \fi
    \lstset{ language=ocaml }
    \lstnewenvironment{mlisting}[1][]{\lstset{inputencoding=latin1, language=ocaml,#1}%
    }{%
    }

    \lstdefinelanguage{csharp}
      { basicstyle=\ttfamily   % Вот тут надо стиль ставить, а не у идентификаторов
      , identifierstyle=\ttfamily
      , sensitive=true
      , keywords={ new, var, in, foreach }
      , keywordstyle=\ttfamily\bfseries\color{MyPurple}
      , commentstyle=\color{eclipseGreen}
      , morecomment=[f][\color{eclipseGreen}][0]{//},
      }


% https://github.com/cansik/kotlin-latex-listing
\lstdefinelanguage{Kotlin}{
  comment=[l]{//},
  commentstyle={\color{gray}\ttfamily},
  emph={filter, first, firstOrNull, forEach, lazy, map, mapNotNull, println},
  emphstyle={\color{OrangeRed}},
  identifierstyle=\color{black},
  keywords={!in, !is, abstract, actual, annotation, as, as?, break, by, catch, class, companion, const, constructor, continue, crossinline, data, delegate, do, dynamic, else, enum, expect, external, false, field, file, final, finally, for, fun, get, if, import, in, infix, init, inline, inner, interface, internal, is, lateinit, noinline, null, object, open, operator, out, override, package, param, private, property, protected, public, receiveris, reified, return, return@, sealed, set, setparam, super, suspend, tailrec, this, throw, true, try, typealias, typeof, val, var, vararg, when, where, while},
  keywordstyle={\color{MyPurple}\bfseries},
  morecomment=[s]{/*}{*/},
  morestring=[b]",
  morestring=[s]{"""*}{*"""},
  ndkeywords={@Deprecated, @JvmField, @JvmName, @JvmOverloads, @JvmStatic, @JvmSynthetic, Array, Byte, Double, Float, Int, Integer, Iterable, Long, Runnable, Short, String, Any, Unit, Nothing},
  ndkeywordstyle={\color{MyPurple}\bfseries},
  sensitive=true,
  stringstyle={\color{ForestGreen}\ttfamily},
}

    %\def\mlinline[1]{\lstinline[langauge=ocaml]{#1}} % is not possible

    \lstdefinelanguage{Kotlin}{
      comment=[l]{//},
      commentstyle={\color{gray}\ttfamily},
      emph={filter, first, firstOrNull, forEach, lazy, map, mapNotNull, println},
      emphstyle={\color{blue}},
      identifierstyle=\color{black},
      keywords={!in, !is, abstract, actual, annotation, as, as?, break, by, catch, class, companion, const, constructor, continue, crossinline, data, delegate, do, dynamic, else, enum, expect, external, false, field, file, final, finally, for, fun, get, if, import, in, infix, init, inline, inner, interface, internal, is, lateinit, noinline, null, object, open, operator, out, override, package, param, private, property, protected, public, receiveris, reified, return, return@, sealed, set, setparam, super, suspend, tailrec, this, throw, true, try, typealias, typeof, val, var, vararg, when, where, while},
      keywordstyle={\color{blue}\bfseries},
      morecomment=[s]{/*}{*/},
      morestring=[b]",
      morestring=[s]{"""*}{*"""},
      ndkeywords={@Deprecated, @JvmField, @JvmName, @JvmOverloads, @JvmStatic, @JvmSynthetic, Array, Byte, Double, Float, Int, Integer, Iterable, Long, Runnable, Short, String, Any, Unit, Nothing},
      ndkeywordstyle={\color{blue}\bfseries},
      sensitive=true,
      stringstyle={\color{ForestGreen}\ttfamily},
    }
    \ifpdftex
        \usepackage{etoolbox}
        \expandafter\patchcmd\csname \string\lstinline\endcsname{%
            \leavevmode
            \bgroup
        }{%
            \leavevmode
            \ifmmode\hbox\fi
            \bgroup
        }{}{%
            \typeout{Patching of \string\lstinline\space failed!}%
        }
    \fi
\fi

\ifminted
    \usepackage[cache=true]{minted}
    \usemintedstyle{perldoc}
    \def\hsinline{\mintinline{haskell}}
    \def\mlinline{\mintinline[escapeinside=||]{ocaml}}

%\def\hsinline{\mintinline{haskell}}
%\def\inline{\hsinline}
\fi
